\documentclass[12pt, oneside]{article}
 
% PKG : police du rapport 
\usepackage[sfdefault]{roboto} 
\usepackage[T1]{fontenc}
\usepackage[utf8]{inputenc}

%%%%%%%%%%%%%%%%%%%%%%%%%%%%%%%%%%%%%%%%%%%%%%%%%%%%%%%%%%%%%%%%%%%%%%%%
%%%%%%%%%%%%%%%%%%%%%%%%%%% ZONE A CHANGER %%%%%%%%%%%%%%%%%%%%%%%%%%%%%
%%%%%%%%%%%%%%%%%%%%%%%%%%%%%%%%%%%%%%%%%%%%%%%%%%%%%%%%%%%%%%%%%%%%%%%%

\newcommand{\type}{05/06/2018}	% Date
\newcommand{\auteur}{\item Jérémy Martin
				      \item Nicolas Fontaine}	% Auteur du compte-rendu
			 
%%%%%%%%%%%%%%%%%%%%%%%%%%%%%%%%%%%%%%%%%%%%%%%%%%%%%%%%%%%%%%%%%%%%%%%%
%%%%%%%%%%%%%%%%%%%%%%%%%% NE PAS Y TOUCHER %%%%%%%%%%%%%%%%%%%%%%%%%%%%
%%%%%%%%%%%%%%%%%%%%%%%%%%%%%%%%%%%%%%%%%%%%%%%%%%%%%%%%%%%%%%%%%%%%%%%%
\newcommand{\titre}{InMoov - Facial Recognition}

%Table des matière intéractif et URL
\usepackage{hyperref}
 
\usepackage[head=30pt,foot=30pt,top=3cm, bottom=3.5cm, outer=2cm, inner=2cm]{geometry}

% PKG : images
\usepackage{graphicx}

% PKG : couleur
\usepackage{xcolor}

% PKG : mise en page
\usepackage{lipsum}

% PKG : item 
\usepackage{enumitem}
\usepackage{pifont}

% PKG : position image avec H en parametre 
\usepackage{float}

% PKG : header et footer
\usepackage{lastpage} 
\usepackage{fancyhdr}
\pagestyle{fancy}

% PKG : pour avoir les dates en francais 
\usepackage[english,francais]{babel}

\usepackage{scrextend}

\usepackage{titlesec}
\newcommand{\sectionbreak}{\clearpage}


\fancypagestyle{style}{
\fancyhf{}
\rhead{\includegraphics[width=4cm]{Images/logo_InMoov.png}}
\chead{\color{gris} \textbf{\titre\\}}
\lhead{\includegraphics[width=5cm]{Images/logo_DaVinciBot.png}}
\cfoot{\includegraphics[width=\linewidth]{img_template/footer_titre.PNG}}
\rfoot{Page \thepage\ sur \pageref{LastPage}}
}

% ---- COLOR ---- %
\definecolor{gris}{rgb}{0.36,0.36,0.36}
\definecolor{bleu}{rgb}{0.1137,0.4941,0.5921}
\definecolor{gristitle}{rgb}{0.95,0.95,0.95}

\hypersetup{%
  colorlinks = true,
  linkcolor  = black,
  urlcolor = bleu 
}
%%%%%%%%%%%%%%%%%%%%%%%%%%%%%%%%%%%
% ---- Code dans le document ---- %
%%%%%%%%%%%%%%%%%%%%%%%%%%%%%%%%%%%
\usepackage{listings}

\lstset{
  basicstyle=\small\color{bleu},
  commentstyle=\color{gray},
}

%%%%%%%%%%%%%%%%%%%%%%%%%%%%%%%%%%%%%
\setlength{\parskip}{1em}
\renewcommand{\baselinestretch}{1.3}

%%%%%%%%%%%%%%%%%%%%%%%%%%%%%%%%%%%%%%%%%%%%%%%%%%%%%%%%%%%%%%%%%%%%%%%%%
%%%%%%%%%%%%%%%%%%%%%%%%%%%% PAGE DE GARDE %%%%%%%%%%%%%%%%%%%%%%%%%%%%%%
%%%%%%%%%%%%%%%%%%%%%%%%%%%%%%%%%%%%%%%%%%%%%%%%%%%%%%%%%%%%%%%%%%%%%%%%%

\begin{document}
\renewcommand{\contentsname}{Table of Contents}
\pagestyle{style}

\begin{center}
\begin{minipage}{0.75\linewidth}
\begin{center}
\centering
   
\vspace{3cm}
\colorbox{gristitle}{
\begin{minipage}{\textwidth}
    \begin{center}
    	\vspace{0.5cm}
    	%% Titre et sous titre 
    	{\color{bleu} \uppercase{\Huge {\titre}}}
    	\vspace{0.25cm}\linebreak
		\par \color{gris} {\Large \type}	
	\end{center}    
\end{minipage}    
}

\vspace{2.5cm}

% Description de Davincibot
\color{bleu} {\Large DaVinciBot \color{gris} : Robotic association of pôle Léonard de Vinci  \par}
\end{center}

\vspace{1.5cm}
% Rapport redigé 
  
%Auteur
\color{gris} {\Large \textbf{Document written by :}
\large
\begin{addmargin}[1em]{1em}
    \begin{itemize}[label= , noitemsep]
    	\auteur
    \end{itemize}
\end{addmargin}}
    
%Degree
\vspace{2.5cm}
%Date
\begin{center}
    {\large Last modification : \today \\ Document made on \LaTeX }
\end{center}
\end{minipage}
\end{center}

\newpage


%%%%%%%%%%%%%%%%%%%%%%%%%%%%%%%%%%%%%%%%%%%%%%%%%%%%%%%%%%%%%%%%%%%%%%%%%
%%%%%%%%%%%%%%%%%%%%%%%%%%%%%%%% CONTENU %%%%%%%%%%%%%%%%%%%%%%%%%%%%%%%%
%%%%%%%%%%%%%%%%%%%%%%%%%%%%%%%%%%%%%%%%%%%%%%%%%%%%%%%%%%%%%%%%%%%%%%%%%

\tableofcontents
\newpage

\section{Introduction}

\paragraph{DaVinciBot,} the robotic association of pôle Léonard De Vinci has developped a facial recognition technology for the InMooV robot.
Four students  in their fourth year of ESILV, engineering school in Paris-La Défense, realized this project :
 
\begin{itemize}
	\item Jérémy Martin
	\item Thomas Guillaume
	\item Chloé Mezouar 
	\item Aria Ekhteraei 
\end{itemize}

Their goal was to give to the InMooV robot the possibilitiy to recognize a person in front of him and recognize the emotion of this person to adapt his own behaviour. \\
In this document, we will present you how to set up this facial recognition. \\
You can find more informations on :\url{http://davincibot.org/inmoov-robot-cognitif/}


\section{Facial recognition}
\subsection{Prerequisites}

To use the facial recognition, you will need several prerequisites :
\begin{itemize}
	\item A camera
	\item A Linux Computer (with Ubuntu preferably)
	\item Python 3.6 installed
\end{itemize}

\subsection{Set up}

First of all, to avoid a lot of problems, it's better to dowload the Python IDE : Thonny. For that, open the Linux interface(Ctr + Alt + T) and write this next command :
\begin{lstlisting}[language=bash]
  $ pip3 install thonny
\end{lstlisting}

Once the IDE installed, open it with this command :
\begin{lstlisting}[language=bash]
  $ thonny
\end{lstlisting}

Now, we will install the libraries that we need. Go on "Tools" then "Manage packages" and install the next packages (it can take some time) :
\begin{itemize}
	\item[•] opencv-python
	\item[•] numpy
	\item[•] scipy
	\item[•] scikit-image (it's not a problem if an error message appear)
 	\item[•] dlib
	\item[•] face\_ recognition
	\item[•] easygui
\end{itemize}

Then, Create a new file in your computer. \\
In this file, put the python script "Face Recognition V3.py" that you can download on this GitHub : \href{https://github.com/MadScientistHK/Pi2_Face_Recognition}{Script python}\\
Then, in this same file, create a new file that you are going to name "Pi2".\\
And inside this file, create an other file that you are going to name "tmp\_ dataset"
\vspace{1cm}
\\
Now, we can get back on the Thonny IDE and open the "Face Recognition V3.py" script and run it.


\subsection{Use}
Once the script run, a window should appear with the footage of yopur computer camera. Every face that goes in front of it is detected with a blue square around it and the name of the person or "unknow" if the person is not in the database.\\
To add a new face there are two options \\
You show your unknow face for several second in front of the camera and wait the "Add message" and floow the instructions.\\
Or you can directly take a picture of the new face and put it in the "tmp\_ dataset" file. The name of the picture is the name of that will appear on the blue square.

\paragraph{WARNING} You can't add two pictures with the same name and it must be only one face on each photos !
\vspace{1cm}

\newpage
Now, if you want to use an extern camera, you just need to replace the "0" by a "1" in the next command :
\begin{lstlisting}[language=bash]
  cap = videocapture(0) 
\end{lstlisting}

\section{Conclusion}
Thank you for reading this tutorial. For more informations on our others InMoov projects, you can go on the DaVinciBot website : \href{http://davincibot.org/inmoov/}{DaVinciBot - InMoov} \\
All the best !

\vspace{2cm}
\begin{center}
	\includegraphics[width = 10cm]{Images/DaVinciBot_Square.png}
\end{center}

\end{document}